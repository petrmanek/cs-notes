\section{Ramseyova teorie}
\begin{t_example}
  Na úvod do Ramseyovy teorie začneme již tradičním příkladem. Mějme večírek, na který pozveme 6 hostů. Tvrdíme, že mezi nimi vždy najdeme trojici lidí, kteří se navzájem znají, nebo trojici lidí, kteří se navzájem určitě neznají. Jak na to přijdeme?

  Zkonstruujeme graf večírku, kde každý host bude jedním vrcholem a každé dva známé spojíme hranou. Tvrzení je ekvivalentní s hledáním kliky nebo nezávislé množiny na třech vrcholech. Podle Ramseyovy věty (viz níže) jednu z nich najdeme na každém grafu velikosti 6 a větší.
\end{t_example}

\begin{t_definition}
  V grafu $G$ definujeme klikovost $\omega(G)$ jako počet vrcholů $n$ největšího úplného grafu $K_n$, který je v $G$ jako podgraf.
\end{t_definition}

\begin{t_definition}
  V grafu $G=(V,E)$ je množina vrcholů $I\subseteq V$ považována za nezávislou, pokud pro každé dva vrcholy v $I$ neexistuje v $E$ hrana.
\end{t_definition}

\begin{t_definition}
  V grafu $G$ definujeme nezávislost $\alpha(G)$ jako velikost největší nezávislé množiny.
\end{t_definition}

\begin{t_theorem}[Ramseyova dvoubarevná]
  Nechť $k,l\in\mathbb{N}$, potom existuje $r\in\mathbb{N}$ takové, že ve všech grafech s alespoň $r$ vrcholy, platí $\omega(G)\geq k$ nebo $\alpha(G)\geq l$.
\end{t_theorem}

\begin{t_remark}
  Číslu $r$ pro daná $k, l$ se říká Ramseyovo číslo, značíme jej $R(k, l)$.
\end{t_remark}

\begin{t_proof}
  Ukážeme horní odhad Ramseyova čísla, tedy takovou hodnotu, že pro každý větší graf Ramseyova vlastnost určitě platí. Tuto hodnotu označíme jako funkci $r(k,l)$ a pokusíme se jí přesně určit. Zvolíme libovolný graf $G$ na $r(k,l)$ vrcholech. V grafu zafixujeme vrchol $v$ a podíváme se na ostatní vrcholy. Pozorujeme, že je můžeme rozdělit do dvou množin podle incidence s $v$. Nechť množina $A$ obsahuje vrcholy incidentní s $v$ a $B$ všechny ostatní vrcholy.
  
  Aby platila Ramseyova vlastnost, musí v $G$ existovat úplný podgraf na $k$ vrcholech nebo nezávislá množina na $l$ vrcholech. Pokud se v množinách $A,B$ nacházejí, máme vyhráno, předpokládejme tedy, že tam nejsou. Množina $A$ stále může tvořit $k$-kliku spolu s vrcholem $v$, pokud v $A$ je úplný podgraf na $k-1$. Podobně může množina $B$ s vrcholem $v$ tvořit nezávislou množinu velikosti $l$, musí v ní ale být nezávislá množina velikosti $l-1$. Získáváme tedy rekurentní požadavek na $r(k,l)$.
  \begin{align*}
    r(k,l)=r(k-1,l)+r(k,l-1)+1
  \end{align*}

  Indukcí můžeme ověřit, že explicitně $r(k,l)=\binom{k+l+2}{k-1}$.
\end{t_proof}

Nyní drobně změníme úhel pohledu o epsilon. Do současné chvíle nás v grafech zajímaly jen kliky a nezávislé množiny, od teď budeme uvažovat o obarvení úplných grafů. Sluší se však poznamenat, že se jedná o úplně ekvivalentní terminologii, neboť každý graf $(V,E)$ můžeme zakódovat do úplného grafu na $|V|$ vrcholech a dvoubarevného obarvení hran $c:\binom{V}{2}\rightarrow\{1,2\}$, které přiřadí jednu barvu všem hranám v $E$ a druhou barvu všem ostatním hranám.  

\begin{t_definition}
  Nechť $f:A\rightarrow B$ je funkce, $A,B,A^\prime$ jsou množiny takové, že $A^\prime\subseteq A$. Potom $f\restriction A^\prime: A^\prime\rightarrow B$ je funkce taková, že $\forall a\in A^\prime:f(a)=(f\restriction A^\prime)(a)$. Funkci $f\restriction A^\prime$ říkáme omezení (restrikce) $f$ na množinu $A^\prime$.
\end{t_definition}

\begin{t_theorem}[Ramseyova vícebarevná spočetná]
  Pro každé $t\in\mathbb{N}$ (počet barev) a obarvení $c:\binom{\mathbb{N}}{2}\rightarrow\{1,2,\dots t\}$ existuje nekonečná množina $A\subseteq\mathbb{N}$, taková že $c\restriction\binom{A}{2}$ je konstantní.
\end{t_theorem}

\begin{t_proof}[nekonečnou indukcí]
  Postupným rozebíráním množiny $\mathbb{N}$ dojdeme až k hledané množině $A$, označíme tedy $A_1=\mathbb{N}$ a nastavíme $i=1$.

  V množině $A_i$ zafixujeme libovolný vrchol $v_i$, zbylé vrcholy $v\in A_i\setminus\{v_i\}$ rozdělíme do množin $B_i^1, B_i^2,\dots B_i^t$ podle barvy $c(\{v_i, v\})$. Víme, že množina $A_i$ je nekonečná, proto alespoň jedna z uvedených stejnobarevných podmnožin musí být také nekonečná. Tuto množinu označíme za $A_{i+1}$ a pokračujeme v indukci.

  Nemáme zaručeno, že pokaždé vybereme podmnožinu stejné barvy, podle které jsme vybrali předchozí množinu. Protože je ale indukce nekonečná, alespoň jednu barvu musíme zvolit nekonečněkrát. Vrcholy $v_i$, které odpovídají této barvě, jsou vrcholy hledané množiny $A$.
\end{t_proof}

\begin{t_theorem}[Ramseyova vícebarevná konečná]
  Nechť $k\in\mathbb{N}$ (požadovaná velikost), potom existuje $r\in\mathbb{N}$ takové, že pro všechna $n\geq r$ a $t\in\mathbb{N}$, $t$-barevná obarvení $c:\binom{\{1,2,\dots n\}}{2}\rightarrow$ najdeme množinu $A\in\binom{\{1,2,\dots n\}}{2}$ takovou, že $c\restriction A$ je konstantní.
\end{t_theorem}
\begin{t_remark}
  Jednodušeji řečeno, pro každé $k$ existuje $r$ takové, že všechny úplné libovolně obarvené grafy na alespoň $r$ vrcholech obsahují obsahují jednobarevnou kliku velikosti $k$ jako podgraf.
\end{t_remark}

\begin{t_proof}
  K důkazu využijeme stejný postup jako u nekonečné varianty. Metodou nekonečného stromu pro dané obarvení vytvoříme posloupnost množin $A_1, A_2,\dots A_d$ a vrcholů $v_1, v_2,\dots v_d$. Stačí nám, když v této posloupnosti bude stejná barva zastoupena $k$-krát. V nejhorším případě se barvy budou střídat a $d\leq tk$.

  Jakmile získáme $k$ zastoupení téže barvy, už máme vyhráno. Zbývá určit velikost $r$. Strom bude mít nejvýše hloubku $tk$, každý vrchol má právě $t$ potomků. Můžeme tedy nejmenší počet vrcholů $r$ odhadnout zespoda jako $\Omega(t^{tk})$. 
\end{t_proof}

\begin{t_theorem}[Ramseyova vícebarevná spočetná pro $p$-tice]
  Pro každé $t\in\mathbb{N}$ (počet barev), $p\in\mathbb{N}$ (velikost $p$-tice) a obarvení $c:\binom{\mathbb{N}}{p}\rightarrow\{1,2,\dots t\}$ existuje nekonečná množina $A\subseteq\mathbb{N}$, taková že $c\restriction\binom{A}{p}$ je konstantní.
\end{t_theorem}

\begin{t_proof}[indukcí podle $p$]
  Použijeme opět stejnou stromovou konstrukci jako u předchozích důkazů. Pro $p=2$ jsme již větu dokázali, zaměříme se tedy na indukční krok. V něm předpokládáme, že pro $p-1$ naše tvrzení platí. Změníme indukci na stromu tak, že v každém jeho vrcholu vytvoříme nové obarvení $(p-1)$-tic $c_i^\prime:\binom{\mathbb{A_i\setminus\{v_i\}}}{p-1}\rightarrow\{1,2,\dots t\}$ takové, že $c_i^\prime(q)=c(q,v_i)$. Na toto obarvení už můžeme použít indukční předpoklad, získáme nekonečnou množinu $C$, na které je $c_i^\prime$ konstantní. A ze způsobu, jak jsme si $c_i^\prime$ zadefinovali, víme, že také všechny $p$-tice složené z $(p-1)$-tic z $C$ a $v_i$ mají konstantní barvu. Množina $C$ je nekonečně velká a konstantně barevná vůči $v_i$, můžeme jí tedy označit $A_{i+1}$ a pokračovat v indukci na stromu.
\end{t_proof}

\begin{t_theorem}[Ramseyova vícebarevná konečná pro $p$-tice]
  Nechť $k\in\mathbb{N}$ (požadovaná velikost) a $p\in\mathbb{N}$ (velikost $p$-tice), potom existuje $r\in\mathbb{N}$ takové, že pro všechna $n\geq r$ a $t\in\mathbb{N}$, $t$-barevná obarvení $c:\binom{\{1,2,\dots n\}}{p}\rightarrow$ najdeme množinu $A\in\binom{\{1,2,\dots n\}}{p}$ takovou, že $c\restriction A$ je konstantní.
\end{t_theorem}

\begin{t_proof}
  Dokážeme tvrzení: "Nekonečná verze Ramseyovy věty pro $p$-tice implikuje konečnou." Dokazovat ale budeme jeho obměnu, tedy: "Pokud neplatí konečná verze Ramseyovy věty pro $p$-tice, neplatí ani nekonečná verze."

  Pokud neplatí nekonečná verze, existuje $p$ (velikost $p$-tice), $t$ (počet barev), nekonečně mnoho $n$ a špatné obarvení $c:\binom{\{1,2,\dots n\}}{p}\rightarrow \{1,2.\dots t\}$ takové, že pro všechny množiny $A\subseteq\{1,2,\dots n\}$ není $c\restriction\binom{A}{p}$ konstantní. Vytvoříme množiny $Š_1, Š_2,\dots Š_n$ všech špatných obarvení na $\{1,2,\dots n\}$. Protože existuje $c$, víme, že $Š_n$ je neprázdná. Nyní zadefinujeme operaci "zkrácení" obarvení $d$ tak, že $z(d)=d\restriction\binom{\{1,2,\dots n-1\}}{p}$. Pozorujeme, že zkrácení každého špatného obarvení je samo špatným obarvením. Formálněji, $\forall c\in Š_n:z(c)\in Š_{n-1}$. Všechny $Š_i$ tedy jsou rovněž neprázdné.

  Zkonstruujeme graf $T$, kde vrcholy budou všechna špatná obarvení. Hrany povedou mezi obarveními $d,e$, pro které platí $e=z(d)$. Pozorujeme, že $T$ je strom s nekonečným počtem vrcholů, každý vrchol však má konečný stupeň. To implikuje, že existuje nekonečná cesta z kořene dolů, tedy posloupnost $c_0,c_1,c_2,\dots$ taková, že $c_i=z(c_{i+1})$ a $c_i\in Š_i$. Položíme $c_\infty=\bigcup_{i\geq 0} c_i$ a zamyslíme se. Toto zobrazení je přesně špatným obarvením $\binom{\mathbb{N}}{p}$, které vyvrátí spočetnou vícebarevnou Ramseyovu větu pro $p$-tice.
\end{t_proof}
