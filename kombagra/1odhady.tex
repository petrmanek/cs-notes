\section{Kombinatorické odhady}
V této části se budeme zabývat odhady veličin často používaných při řešení kombinatorických problémů. Na začátek ale zadefinujeme dva klíčové pojmy, bez kterých se neobejdeme.

\begin{t_definition}
  Pro $n\in\mathbb{N}$ definujeme faktoriál jako $n!=1\cdot 2\cdot 3\dots n$. Speciálně, nechť je $0!$ roven 1.
\end{t_definition}

\begin{t_definition}
  Pro $n,k\in\mathbb{N}, k\leq n$ definujeme kombinační číslo $\binom{n}{k}=\frac{n!}{k!(n-k)!}$. Kombinačnímu číslu $\binom{n}{k}$ také říkáme binomický koeficient, ve výrazech jej čteme \textit{$n$ nad $k$}.
\end{t_definition}

\begin{t_observation}
  Platí následující identity: $\binom{n}{k}=\binom{n}{n-k}$, $\binom{n}{0}=\binom{n}{n}=1$, $\binom{n}{1}=\binom{n}{n-1}=n$
\end{t_observation}

\begin{t_theorem}[odhad faktoriálu]
  Nechť $n\in\mathbb{N}$, potom platí:
  \begin{align*}
    e\left(\frac{n}{e}\right)^n\leq n! \leq en\left(\frac{n}{e}\right)^n
  \end{align*}
\end{t_theorem}

\begin{t_proof}[Důkaz (indukcí přes $n$)]
  Pro $n=1$ je rovnost zredukována na $1\leq 1\leq 1$. Můžeme tedy přikročit k indukčnímu kroku. V něm dokážeme každou nerovnost zvlášť. Nejprve se podíváme na horní mez. Z indukčního předpokladu vyplývá následující nerovnost.
  \begin{align*}
    n! = n(n-1)! &\leq en(n-1)\left(\frac{n-1}{e}\right)^{n-1}
  \end{align*}
  Tuto nerovnost můžeme dále ekvivalentně upravovat.
  \begin{align*}
    en(n-1)\left(\frac{n-1}{e}\right)^{n-1}
    &= en\frac{(n-1)^n}{e^{n-1}}\\
    &= en\frac{(n-1)^n}{e^{n-1}}\cdot\frac{n^n}{n^n}\cdot\frac{e}{e}\\
    &= en\left(\frac{n}{e}\right)^n\cdot e\left(\frac{n-1}{n}\right)^n\\
    &= en\left(\frac{n}{e}\right)^n\cdot e\left(1-\frac{1}{n}\right)^n
  \end{align*}
  Nyní použijeme fakt, že $\left(1-\frac{1}{n}\right)^n\leq \frac{1}{e}$, a nerovnost doupravíme do požadované formy.
  \begin{align*}
    en\left(\frac{n}{e}\right)^n\cdot e\left(1-\frac{1}{n}\right)^n
    &\leq en\left(\frac{n}{e}\right)^n\cdot \frac{e}{e}\\
    &= en\left(\frac{n}{e}\right)^n
  \end{align*}
  Tím je důkaz pravé nerovnosti hotov. Nyní provedeme analogickou úvahu i pro levou nerovnost, začneme opět u indukčního předpokladu.
  \begin{align*}
    n! = n(n-1)! &\geq en\left(\frac{n-1}{e}\right)^{n-1}
  \end{align*}
  Provedeme podobné ekvivalentní úpravy jako u minulého případu.
  \begin{align*}
    en\left(\frac{n-1}{e}\right)^{n-1}
    &= en\left(\frac{n-1}{e}\right)^{n-1}\cdot\frac{n^n}{n^n}\cdot\frac{e}{e}\\
    &= e\left(\frac{n-1}{n}\right)^{n-1}\cdot\frac{n^ne}{e^{n}}\\
    &= \frac{e^2\left(\frac{n}{e}\right)^n}{\left(\frac{n}{n-1}\right)^{n-1}}\\
    &= \frac{e^2\left(\frac{n}{e}\right)^n}{\left(1+\frac{1}{n-1}\right)^{n-1}}
  \end{align*}
  Přišel čas na připomenutí faktu, že $\left(1-\frac{1}{n}\right)^n\leq \frac{1}{e}$. Jelikož je však část výrazu ve jmenovateli, nesmíme zapomenout nerovnost otočit.
  \begin{align*}
    \frac{e^2\left(\frac{n}{e}\right)^n}{\left(1+\frac{1}{n-1}\right)^{n-1}}
    &\geq \frac{e^2\left(\frac{n}{e}\right)^n}{\left(e^{\frac{1}{n-1}}\right)^{n-1}}\\
    &= \frac{e^2\left(\frac{n}{e}\right)^n}{e}\\
    &= e\left(\frac{n}{e}\right)^n
  \end{align*}
  Podařilo se nám tedy úspěšně dokázat horní i dolní mez pro faktoriál. Důkaz je dokončen.
\end{t_proof}

Ukazuje se, že tento odhad bohužel není příliš těsný. Lepší aproximaci poskytuje takzvaná Stirlingova formule. Její důkaz ale vyžaduje netriviální využití gamma funkce, proto jej zde vynecháme.
\begin{t_fact}[Stirlingova formule]
  Nechť $n\in\mathbb{N}$, potom platí:
  \begin{align*}
    n! \approx \sqrt{2\pi n}\left(\frac{n}{e}\right)^n
  \end{align*}  
\end{t_fact}

Stirlingův odhad faktoriálu je ve skutečnosti natolik těsný, že jej můžeme využít k velmi dobrému odhadu kombinačních čísel. Stačí jej substituovat do vzorce $\frac{n!}{k!(n-k)!}$. Jediná slabina tohoto odhadu je, že je jako matematický výraz příliš komplikovaný. Proto si ukážeme odhad, který je o epsilon horší, ale zároveň výrazně jednodušší.

\begin{t_theorem}[odhad kombinačního čísla]
  Nechť $n,k\in\mathbb{N}$, $k\leq n$, potom platí:
  \begin{align*}
    \left(\frac{n}{k}\right)^k\leq \binom{n}{k}\leq \frac{1}{e}\cdot\left(\frac{en}{k}\right)^k
  \end{align*}
\end{t_theorem}
\begin{t_proof}
  \begin{align*}
    \binom{n}{k}=\frac{n(n-1)\dots(n-k+1)}{k!}
    \leq \frac{n^k}{k!}
    \leq \frac{n^k}{e\left(\frac{k}{e}\right)^k}
    = \frac{1}{e}\cdot\left(\frac{en}{k}\right)^k
  \end{align*}
  
  Důkaz dolního odhadu bude trochu komplikovanější, budeme k němu muset použít oba odhady faktoriálu, které jsme dokázali dříve. Protože odhadujeme celý zlomek, musíme ve jmenovateli opět použít opačný odhad.
  \begin{align*}
    \binom{n}{k}
    = \frac{n!}{k!(n-k)!}
    \geq \frac{en\left(\frac{n}{e}\right)^n}{e\left(\frac{k}{e}\right)^ke\left(\frac{n-k}{e}\right)^{n-k}}
    = \frac{n^{n+1}}{k^k(n-k)^{n-k}}
  \end{align*}
  
  Hodně mocnin se nám podařilo zkrátit, nyní musíme postupovat ekvivalentními úpravami.
  \begin{align*}
    \frac{n^{n+1}}{k^k(n-k)^{n-k}}
    &\stackrel{?}{\geq} \left(\frac{n}{k}\right)^k=\frac{n^k}{k^k}\\
    \frac{n^{n+1}}{(n-k)^{n-k}}
    &\stackrel{?}{\geq} n^k\\
    n\cdot \frac{n^{n-k}}{(n-k)^{n-k}}
    &\stackrel{?}{\geq} 1\\
    n\cdot\underbrace{\left(\frac{n}{n-k}\right)^{n-k}}_{\geq 1}
    &\geq 1
  \end{align*}
\end{t_proof}

Z uvedeného odhadu vyplývá speciální případ pro centrální binomické koeficienty.
\begin{t_corollary}
  Nechť $n\in\mathbb{N}$, potom platí:
  \begin{align*}
    2^n\leq \binom{2n}{n}\leq 2^ne^{n-1}
  \end{align*}
\end{t_corollary}

\begin{t_exercise}
  \item Porovnejte $\binom{80}{20}$, $\binom{90}{20}$, $\binom{90}{80}$, $\binom{80}{60}$, $\binom{90}{30}$, $\binom{80}{10}$.
  \item Porovnejte $n!$, $\binom{2n}{n}$, $\binom{2n}{n-1}$, $n^{\sqrt{n}}$, $\sqrt{n}^n$.
  \item Spočítejte, kolik nulových cifer je na konci dekadického zápisu čísla $12723!$.
  \item Rozhodněte, která z následujících funkcí roste asymptoticky rychleji:
  \\$f(n)=\binom{10n}{\lceil\sqrt{n}\rceil}$ nebo $g(n)=n^\sqrt{n}$.
  \item Rozhodněte, která z následujících funkcí roste asymptoticky rychleji. Funkce $f(n)$ označuje počet podmnožin velikosti $n$ množiny obsahující $2n$ prvků. Funkce $g(n)$ označuje počet podmnožin velikosti nejvýše $0,999n$ množiny obsahující $2n$ prvků.
  \item Označme $r(n)$ počet rovinných grafů na množině vrcholů $1\dots n$ a $g(n)$ na stejné množině vrcholů. Dokažte, že pro každé dostatečně velké $n$ platí:
  \begin{align*}
    r(\lfloor n^{1,999} \rfloor)<g(n)<r(n^2)
  \end{align*}
  \item Kolik z přirozených čísel $1\dots 420$ není dělitelných 6, 17 ani 42?
  \item Kolik různých (ne nutně smysluplných) slov vznikne permutací písmen slova \textit{abrakadabra}? Slovem myslíme jen posloupnost znaků.
\end{t_exercise}
