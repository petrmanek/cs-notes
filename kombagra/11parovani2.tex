\section{Párování podruhé}
\begin{t_definition}
  Buď $G=(V,E)$ graf. Párování v $G$ je množina hran $M \subseteq E$ taková, že každý vrchol sousedí nejvýše s 1 hranou $M$.
\end{t_definition}

\begin{t_definition}
  Volný vrchol v párování $M$ je vrchol, který není v žádné párovací hraně, tedy $\forall e \in M: \; v \not\in e$.
\end{t_definition}

\begin{t_definition}
  Párování $M$ je největší (anglicky maximum), pokud neexistuje párování $M^\prime$, kde $|M^\prime| > |M|$.
\end{t_definition}

\begin{t_definition}
  Párování $M$ je maximální (anglicky maximal), pokud neexistuje párování $M^\prime$, kde $M \neq M^\prime$ a $M \subseteq M^\prime$.
\end{t_definition}

\begin{t_definition}
  Párování $M$ je perfektní, pokud nemá žádný volný vrchol.
\end{t_definition}

\begin{t_definition}
  Mějme $G=(V,E)$ graf, $M$ párování v $G$. Střídavá cesta vzhledem k $M$ je cesta v $G$, na níž se střídají párovací a nepárovací hrany.
\end{t_definition}

\begin{t_definition}
  Volná střídavá cesta (VSC) (jinak také zlepšující cesta, neboli augmenting path) je střídavá cesta, která spojuje 2 volné vrcholy.
\end{t_definition}

\begin{t_lemma}
  Buď $G=(V,E)$ graf, $M$ jeho párování. Pak $M$ je největší $\iff$ $M$ nemá VSC.
\end{t_lemma}

\begin{t_proof}
  Implikaci zleva doprava ukážeme jednoduše sporem. Pokud by $M$ mělo VSC, vyměníme v té VSC párovací hrany za nepárovací a naopak. Tím jsme dostali párování, které má o jednu párovací hranu více, což je spor s tím, že $M$ bylo největší párování. \\
  Opačná implikace obměnou: Nechť M není největší. Volme tedy $M^\prime$ párování v $G$ takové, že $|M^\prime| > |M|$. Označme si $N = M \triangle M^\prime$ a uvažme graf $H=(V,N)$. Pozorování: 
  \begin{itemize}
    \item V tomto grafu má každý vrchol stupeň nejvýše 2, což vyplývá z volby množiny hran, tedy každá komponenta $H$ je cesta anebo kružnice. 
    \item Každá kružnice má stejný počet hran z $M$ jako z $M^\prime$. 
    \item Platí $|M^\prime| > |M|$ $\Rightarrow$ existuje komponenta $H$, která má více hran z $M^\prime$ než z $M$. Tato komponenta je však VSC pro $M$.
  \end{itemize}
\end{t_proof}
  Již tedy víme něco o párování, chtěli bychom tedy algoritmus, který nalezne největší párování.
  
\begin{t_observation}
  Stačí algoritmus, který pro zadaný $G$ a jeho párování $M$ najde větší párování, nebo rozhodne, že takové neexistuje.
\end{t_observation}

\begin{t_definition}
  Kytka v grafu $G$ s párováním $M$ je podgraf, který je složen z kružnice $C_i$, která má lichou délku a právě $\lfloor i/2 \rfloor$ párovacích hran, a střídavé cesty sudé délky, která má na svém konci volný vrchol a její druhý konec je vrchol na kružnici $C_i$. Kružnici nazýváme \textit{květ} a cestu \textit{stonek}. Pozorujeme, že z květu vychází nejvýše jedna párovací hrana do stonku. (Kytka nemusí mít stonek.)
\end{t_definition}

\begin{t_definition}
  Kontrakce květu $K$ je operace, kde všechny vrcholy kružnice nahradíme jedním novým vrcholem $V_K$, jehož sousedé jsou sjednocení sousedů všech vrcholů květu $K$. Graf vzniklý kontrakcí květu značíme $G.K$ a jeho párování $M.K$
\end{t_definition}

\begin{t_lemma}$ $
  \begin{enumerate}
    \item $G$ je graf, $K$ květ kytky v $M$. Pak $M$ je největší v $G$ $\iff$ $M.K$ je největší v $G.K$.
    \item Když najdeme v $G.K$ párování $M^\prime$ větší, než $M.K$, tak z něj umíme získat v polynomiálním čase párování $M^\prime$ v grafu $G$, které je větší, než $M$.
  \end{enumerate}
\end{t_lemma}

\begin{algorithm}
\caption{Edmonsův kytičkový (blossom) algoritmus (1963)}
\textbf{Kostra algoritmu:}
\begin{enumerate}
  \item
  Zkus najít VSC anebo kytku v $(G,M)$. 
  \begin{itemize}
    \item Pokud najdeš VSC, tak zlepši $M$.
    \item Pokud najdeš kytku, zavolej se na $(G.K, M.K)$ Pokud $M.K$ lze zlepšit, zlepši $M$, jinak konec.
  \end{itemize}
  \item
  Pokud už není ani VSC ani kytka, konec.
\end{enumerate}

\begin{algorithmic}[1]
  \Require graf $G$, párování $M$.
  \State Vytvoř frontu vrcholů $F\gets\emptyset$, les $L\gets\emptyset$. 
  \State Najdi volné vrcholy, vlož je do $F$ a do $L$ na hladinu $\Theta$.
  \While {$F = \emptyset$}
    \State Vyjmi první vrchol z fronty $F$, přiřaď ho do $v$ a označ $h$ hladinu $v$.
    \If {$h$ lichá}
      \State Zvolím $y$ vrchol spojený s $v$ párovací hranou.
      \If {$y \not\in L$}
        \State Zařaď $y$ do $F$, přidej $y$ do $L$ spolu s $vy$ na hladinu $h+1$.
      \Else
        \State Vrchol $y$ je na hladině $h$, tedy najdu VSC nebo $K$.
      \EndIf
    \Else
      \State Zvolím $Y$ jako množinu vrcholů dosažitelných z $v$ po nepárovací hraně.
      \If {$\exists y \in Y, y \in L : h(y)$ je sudé}
        \State Najdu VSC nebo $K$.
      \Else
        \State $\forall y \in Y$ nepatřící do $L$, zařaď $y$ do $F$, přidej $y$ a $vy$ do $L$ na hladinu $h+1$.
      \EndIf
    \EndIf
  \EndWhile
  \State Pokud se F se vyprázdní bez nalezení VSC nebo kytky, $M$ je největší párování.
\end{algorithmic}
\end{algorithm}
